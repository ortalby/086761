\DeclareMathOperator*{\argmin}{arg\,min}
\DeclareMathOperator*{\argmax}{arg\,max}

%\theoremstyle{definition}
\newtheorem{example}{Example}[section]


\definecolor{vadim}{rgb}{1,0.6,0.7}
\definecolor{yuri}{rgb}{0.6,0.6,1}

\definecolor{marker}{rgb}{1,1,0}

\newcommand{\myputtext}[2]{
	\begin{tikzpicture}
	\node [fill=#2, rounded corners=2pt] {#1};
	\end{tikzpicture}
}
%\newcommand{\VI}[1]{\colorbox{vadim}{VI: #1}}
%\newcommand{\YF}[1]{\colorbox{yuri}{YF:#1}}
\newcommand{\VI}[1]{{\color{blue} #1}}
\newcommand{\YF}[1]{{\color{magenta} #1}}


\newcommand{\hl}[1]{\colorbox{marker}{#1}}

\newcommand{\Eqref}[1]{Eq.~(\ref{#1})}
\newcommand{\Figref}[1]{Fig.~\ref{#1}}


\newcommand{\Zeta}{\mathrm{Z}}
\newcommand{\bydef}{\ensuremath{\overset{def}{=}}}
\newcommand{\maxim}[2]{\ensuremath{\underset{#2}{#1}}}
\newcommand{\fnorm}[1]{\ensuremath{{\parallel}#1{\parallel_{\cal F}}}} % this removes spacing, could be ugly for final version
\newcommand{\nrm}[2]{\ensuremath{{\parallel}{#2}{\parallel_{#1}}}}
\newcommand{\nrmsq}[2]{\ensuremath{{\parallel}{#2}{\parallel^2_{#1}}}}
\newcommand{\expt}[2]{\ensuremath{\underset{{#1}}{\mathbb E}{\{{#2}\}}}}
%\newcommand{\alias}[2]{{#1}\odot{#2}}
\newcommand{\alias}[1]{\ensuremath{\{{#1}\}_{\textbf{aliased}}}}% \xspace}
%\newcommand{\blf}[1]{\ensuremath{b[#1]}}

\newcommand{\prob}[1]{\ensuremath{\mathbb{P}({#1})}}
\newcommand{\blf}[1]{\prob{#1}}


%\usepackage{textcomp,    % for \textlangle and \textrangle macros
%            xspace}
%\newcommand\la{\textlangle\xspace}  % set up short-form macros
%\newcommand\ra{\textrangle\xspace}

%FIXME!
\newcommand\la{\langle\xspace}  % set up short-form macros
\newcommand\ra{\rangle\xspace}

\newcommand*\Let[2]{\State #1 $\gets$ #2}
\algrenewcommand\algorithmicrequire{\textbf{Input:}}
\algrenewcommand\algorithmicensure{\textbf{Input:}}
\algnewcommand{\LineComment}[1]{\State \(\triangleright\) #1}

%new formulation
\newcommand{\priorB}{\ensuremath{\blf{X^-_{k+1}}}\xspace}
\newcommand{\condB}{\blf{X_{k+1} \mid z_{k+1}, \his}\xspace}
\newcommand{\condBi}[1]{\blf{X_{k+1} \mid #1, z_{k+1}, \his}\xspace}
\newcommand{\event}[1]{\ensuremath{A_{#1}}\xspace}

%spaces
\newcommand{\poses}{\ensuremath{{\cal X}}\xspace}
\newcommand{\relpose}{\ensuremath{x^{(rel)}}\xspace}
\newcommand{\relposes}{\ensuremath{{\cal X}^{(rel)}}\xspace}
\newcommand{\observations}{\ensuremath{{\cal Z}}\xspace}
\newcommand{\events}{\ensuremath{\{\event{\mathbb{N}}\}}\xspace}
\newcommand{\controls}{\ensuremath{{\cal U}}\xspace}
\newcommand{\landmarks}{\ensuremath{{\cal L}}\xspace}

\newcommand{\classif}{\ensuremath{{\cal S}}\xspace}
\newcommand{\classes}{\ensuremath{{\cal C}}\xspace}

%history
\newcommand{\his}{\ensuremath{{\cal H}}\xspace}

% history with relative poses
\newcommand{\hisrp}{\ensuremath{{H}}\xspace}

%------------------------------------------------
% Statistical Independence 
% from http://jblevins.org/log/latex-tips
\newcommand\independent{\protect\mathpalette{\protect\independenT}{\perp}}
\def\independenT#1#2{\mathrel{\rlap{$#1#2$}\mkern2mu{#1#2}}}
%------------------------------------------------