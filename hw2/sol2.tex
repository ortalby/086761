\documentclass[a4paper]{scrreprt}

\usepackage[margin=0.1in]{geometry}

\usepackage[english]{babel}
\usepackage[utf8]{inputenc}

\usepackage{color,xcolor,ucs}

%\usepackage{mathptmx}       % selects Times Roman as basic font
%\usepackage{subfig}
\usepackage{floatrow}
\usepackage{tabularx}
\usepackage{float}
\usepackage{amsfonts}
\usepackage{helvet}         % selects Helvetica as sans-serif font
\usepackage{courier}        % selects Courier as typewriter font
\usepackage{type1cm}        % activate if the above 3 fonts are
% not available on your system
\usepackage{amsmath}
\usepackage{mathtools}
% using for \triangleq
% from http://tex.stackexchange.com/questions/151670/latex-symbol-for-leftrightarrow-with-triangleq
\usepackage{amssymb}
%
\usepackage{makeidx}         % allows index generation
\usepackage{comment}         % allows index generation
\usepackage{graphicx}        % standard LaTeX graphics tool
% when including figure files
\usepackage{epstopdf}
\epstopdfsetup{outdir=./figures/}

\usepackage{multicol}        % used for the two-column index
\usepackage[bottom]{footmisc}% places footnotes at page bottom
\usepackage{bm}

\usepackage{cite}
\usepackage{url}

%\usepackage[bookmarks=true]{hyperref}


\usepackage[unicode=true, bookmarks=true,colorlinks=true]{hyperref}
\usepackage{xr-hyper}


%\newfloat{algorithm}{t}{lop}
%\usepackage[linesnumbered, inoutnumbered]{algorithm2e}
\usepackage{algpseudocode,algorithm,algorithmicx}



\usepackage{xspace}
\usepackage{rotating}

\usepackage{tikz}
%\usepackage{pgfplots}
\usepackage{tkz-graph}
\usetikzlibrary{arrows,shapes,shadows,positioning,calc}


\usepackage{graphicx}
%\graphicspath{{figures/}}

\usepackage{threeparttable}
\usepackage{multirow}
%\usepackage[font=scriptsize,labelfont=bf]{caption}

\usepackage{blkarray}

\usepackage{tikz}
\usetikzlibrary{bayesnet}
\usepackage{subcaption}
\usepackage{verbatim}
\graphicspath{{images/}}

%\usepackage{setspace}
%\usepackage{titlesec}
%\titlespacing*{\section}
%{0pt}{5.5ex plus 1ex minus .2ex}{4.3ex plus .2ex}
%\titlespacing*{\subsection}
%{0pt}{5.5ex plus 1ex minus .2ex}{2.3ex plus .2ex}
\DeclareMathOperator*{\argmin}{arg\,min}
\DeclareMathOperator*{\argmax}{arg\,max}

%\theoremstyle{definition}
\newtheorem{example}{Example}[section]


\definecolor{vadim}{rgb}{1,0.6,0.7}
\definecolor{yuri}{rgb}{0.6,0.6,1}

\definecolor{marker}{rgb}{1,1,0}

\newcommand{\myputtext}[2]{
	\begin{tikzpicture}
	\node [fill=#2, rounded corners=2pt] {#1};
	\end{tikzpicture}
}
%\newcommand{\VI}[1]{\colorbox{vadim}{VI: #1}}
%\newcommand{\YF}[1]{\colorbox{yuri}{YF:#1}}
\newcommand{\VI}[1]{{\color{blue} #1}}
\newcommand{\YF}[1]{{\color{magenta} #1}}


\newcommand{\hl}[1]{\colorbox{marker}{#1}}

\newcommand{\Eqref}[1]{Eq.~(\ref{#1})}
\newcommand{\Figref}[1]{Fig.~\ref{#1}}


\newcommand{\Zeta}{\mathrm{Z}}
\newcommand{\bydef}{\ensuremath{\overset{def}{=}}}
\newcommand{\maxim}[2]{\ensuremath{\underset{#2}{#1}}}
\newcommand{\fnorm}[1]{\ensuremath{{\parallel}#1{\parallel_{\cal F}}}} % this removes spacing, could be ugly for final version
\newcommand{\nrm}[2]{\ensuremath{{\parallel}{#2}{\parallel_{#1}}}}
\newcommand{\nrmsq}[2]{\ensuremath{{\parallel}{#2}{\parallel^2_{#1}}}}
\newcommand{\expt}[2]{\ensuremath{\underset{{#1}}{\mathbb E}{\{{#2}\}}}}
%\newcommand{\alias}[2]{{#1}\odot{#2}}
\newcommand{\alias}[1]{\ensuremath{\{{#1}\}_{\textbf{aliased}}}}% \xspace}
%\newcommand{\blf}[1]{\ensuremath{b[#1]}}

\newcommand{\prob}[1]{\ensuremath{\mathbb{P}({#1})}}
\newcommand{\blf}[1]{\prob{#1}}


%\usepackage{textcomp,    % for \textlangle and \textrangle macros
%            xspace}
%\newcommand\la{\textlangle\xspace}  % set up short-form macros
%\newcommand\ra{\textrangle\xspace}

%FIXME!
\newcommand\la{\langle\xspace}  % set up short-form macros
\newcommand\ra{\rangle\xspace}

\newcommand*\Let[2]{\State #1 $\gets$ #2}
\algrenewcommand\algorithmicrequire{\textbf{Input:}}
\algrenewcommand\algorithmicensure{\textbf{Input:}}
\algnewcommand{\LineComment}[1]{\State \(\triangleright\) #1}

%new formulation
\newcommand{\priorB}{\ensuremath{\blf{X^-_{k+1}}}\xspace}
\newcommand{\condB}{\blf{X_{k+1} \mid z_{k+1}, \his}\xspace}
\newcommand{\condBi}[1]{\blf{X_{k+1} \mid #1, z_{k+1}, \his}\xspace}
\newcommand{\event}[1]{\ensuremath{A_{#1}}\xspace}

%spaces
\newcommand{\poses}{\ensuremath{{\cal X}}\xspace}
\newcommand{\relpose}{\ensuremath{x^{(rel)}}\xspace}
\newcommand{\relposes}{\ensuremath{{\cal X}^{(rel)}}\xspace}
\newcommand{\observations}{\ensuremath{{\cal Z}}\xspace}
\newcommand{\events}{\ensuremath{\{\event{\mathbb{N}}\}}\xspace}
\newcommand{\controls}{\ensuremath{{\cal U}}\xspace}
\newcommand{\landmarks}{\ensuremath{{\cal L}}\xspace}

\newcommand{\classif}{\ensuremath{{\cal S}}\xspace}
\newcommand{\classes}{\ensuremath{{\cal C}}\xspace}

%history
\newcommand{\his}{\ensuremath{{\cal H}}\xspace}

% history with relative poses
\newcommand{\hisrp}{\ensuremath{{H}}\xspace}

%------------------------------------------------
% Statistical Independence 
% from http://jblevins.org/log/latex-tips
\newcommand\independent{\protect\mathpalette{\protect\independenT}{\perp}}
\def\independenT#1#2{\mathrel{\rlap{$#1#2$}\mkern2mu{#1#2}}}
%------------------------------------------------

\title{086761 - Homework 2}
\author{Yuri Feldman, 309467801 yurif@cs.technion.ac.il \\
	    Alexander Shender, 328626114 aka.sova@gmail.com }
\begin{document}
\maketitle
\chapter{}
\section{}
Substitute $\Lambda^{-1} = \Sigma$, $\eta = \Sigma^{-1}\mu$:
\begin{gather}
	p(x) = 
	\frac{\exp\left(-\frac{1}{2}\mu^T\Sigma^{-T}\Sigma\Sigma^{-1}\mu\right)}{\sqrt{\det{2\pi\Sigma}}}
	\exp \left(-\frac{1}{2}x^T\Sigma^{-1}x+\mu^T\Sigma^{-1}\right) = \\
	\intertext{$\Sigma^{-1}=\Sigma^{-T}$ since it is symmetric}
	= 
	\frac{\exp\left(-\frac{1}{2}\left(x^T\Sigma^{-1}x - 2\mu^T\Sigma^{-1}  + 
	\mu^T\Sigma^{-1}\mu\right)\right)}{\sqrt{\det{2\pi\Sigma}}} 
	= \\
	= \frac{1}{\sqrt{\det{2\pi\Sigma}}}
	\exp\left\lbrace-\frac{1}{2}\left(x-\mu\right)^T\Sigma^{-1}\left(x-\mu\right)\right\rbrace
\end{gather}
the latter is the original multivariate Gaussian. 


\chapter{}
\section{}
\begin{gather}
	p(x) = N(\hat{x_0}, \Sigma_0) = 
	\frac{1}{\sqrt{\det\left(2\pi\Sigma_0\right)}} 
	e^{-\frac{1}{2}\parallel x-x_0 \parallel^2_{\Sigma_0}} \\
	p(z\mid x) = N(h(x), \Sigma_v) = 
	\frac{1}{\sqrt{\det\left(2\pi\Sigma_v\right)}} 
		e^{-\frac{1}{2}\parallel z-h(x) \parallel^2_{\Sigma_v}}
\end{gather}

\section{}
\begin{gather}
	p(x\mid z_1) = \frac{p(z_1\mid x ) p(x)}{\int_x p(z_1\mid x) p(x)dx}
\end{gather}

\section{}
We focus on the exponent (omitting the $-\frac{1}{2}$ factor) in the above 
product, since all the rest is 
normalization factor (does not depend on $x$). 
\begin{gather}
	\parallel h(x)-z_1 \parallel^2_{_{\Sigma_v}} + \parallel x-\hat{x_0 }
	\parallel_{_{\Sigma_0}}^2 \approx 
	\intertext{linearization around $\hat{x_0}$, H the Jacobian of h at 
	$\hat{x}_0$:}
	\approx \nrmsq{\Sigma_v}{h(\hat{x_0})+ H(x-\hat{x_0})-z_1}
	+\nrmsq{\Sigma_0}{x-\hat{x}_0} = \\
	= x^T\left(H^T\Sigma_v^{-1}H+\Sigma_0^{-1} \right)x
	+2\left((h(\hat{x}_0)-H\hat{x}_0-z_1)^T\Sigma_v^{-1}H-\hat{x}_0^T\Sigma_0^{-1}\right)x
	 +\dots 
	\intertext{equating terms with the exponent of a MVN, we get:}
	\Sigma_1 = \left(H^T\Sigma_v^{-1}H+\Sigma_0^{-1}\right)^{-1}
	\\
	\hat{x}_1 = \Sigma_1
	\left(\Sigma_0^{-1}\hat{x}_0- 
	H^T\Sigma_v^{-1}(h(\hat{x}_0)-H\hat{x}_0-z_1)\right)
%	\nrmsq{}{\Sigma_v^{-\frac{1}{2}}\left(h(\hat{x_0})+ 
%	H(x-\hat{x_0})-z_1\right)}
%	+\nrmsq{}{\Sigma_0^{-\frac{1}{2}}\left(x-\hat{x}_0\right)} 
\end{gather}
Can iteratively use obtained $\hat{x}_1$ as the new linearization point 
(re-calculate jacobian $H$) to improve estimate until convergence (Gauss 
Newton). 

\section{}
\begin{gather}
	p(x\mid z_1, z_2) \propto p(z_1, z_2\mid x) p(x) = 
	\intertext{assume noise $v_1$, $v_2$ independent}
	= p(z_1\mid x) p(z_2\mid x) p(x)
\end{gather}
We again get a Gaussian, can obtain mean and variance by minimizing the 
exponent (again omitting the $-\frac{1}{2}$ factor)
\begin{gather}
	\nrmsq{\Sigma_v}{h(x)-z_2} + 
	\nrmsq{\Sigma_v}{h(x)-z_1}  + \nrmsq{\Sigma_0}{x-\hat{x_0 }}
\end{gather}
Again need to pick a linearization point (e.g. $\hat{x}_1$ is a sensible one), 
denote $\bar{x}$ as in the lecture, 
then
\begin{gather}
	\approx 
		\nrmsq{\Sigma_v}{Hx+ h(\bar{x})-H\bar{x}-z_2} + 
		\nrmsq{\Sigma_v}{Hx +h(\bar{x}) -H\bar{x}-z_1}  + 
		\nrmsq{\Sigma_0}{x-\hat{x_0 }}\\
	\intertext{Following the same derivation as before, it's easy to see that}
	\Sigma_2 = \left(2H^T\Sigma_v^{-1}H+\Sigma_0^{-1}\right)^{-1}
		\\
		\hat{x}_2 = \Sigma_2
		\left(\Sigma_0^{-1}\hat{x}_0- 
		2H^T\Sigma_v^{-1}(h(\hat{x}_0)-H\hat{x}_0-\frac{z_1+z_2}{2})\right)
\end{gather}
And again as in the previous clause, we can re-linearize to get better esimates 
with Gauss-Newton iterations. 

\chapter{}
\section{}
\begin{gather}
	p(x_k\mid x_{k-1}, u_{k-1}) = N(f(x_{k-1}, u_{k-1}), \Sigma_w)
	= \frac{1}{\sqrt{\det\left(2\pi \Sigma_w\right)}} e^{-\frac{1}{2} 
	\nrmsq{_{\Sigma_w}}{x_k-f(x_{k-1},u_{k-1})}}
\end{gather}
\section{}
\begin{gather}
	p(x_1\mid z_1, u_0) = \frac{p(z_1\mid x_1)\int_{x_0}p(x_1\mid x_0, 
	u_0)p(x_0)dx_0}{\int_{x_1, x_0} p(z_1\mid x_1) p(x_1\mid x_0, u_0) 
	p(x_0)dx_1dx_0 } 
\end{gather}
\section{}
%The integrand in the numerator of the previous clause is a multivariate 
%Gaussian over 
%$x_{0:1}$. Marginalization over $x_0$ (the integral) yields a Gaussian over 
%$x_1$, as was seen at the lecture, with parameters - the corresponding 
%elements 
%of the joint mean and covariance. Denote the marginalization result as 
%$p(x_1) = N(\mu_1^-,\Sigma_1^- )$ (in principle, we're back to the situation 
%from the previous question. Still - we develop again)
From previous clause: 
\begin{gather}
	x_1^{MAP} \doteq \argmax_{x_1} p(x_1\mid z_1, u_0) = \argmin_{x_1}
	-\log\left(p(z_1\mid x_1) p(x_1)\right) = \\
	\argmin_{x_1}\, \nrmsq{_{\Sigma_v}}{z_1-h(x_1)} -\log \int_{x_0} p(x_1\mid 
	x_0, u_0) p(x_0) dx_0
\end{gather}
The integrand has the form of a gaussian, however with the left term exponent 
with generally nonlinear dependence on $x_0$: 
\begin{gather}
	p(x_1\mid 
		x_0, u_0) p(x_0)\propto \exp 
		\{-\frac{1}{2}\left(\nrmsq{}{\Sigma_w^{-\frac{1}{2}}\left(x_1-f(x_0,u_0)\right)}+
		\nrmsq{}{\Sigma_0^{-\frac{1}{2}}\left(x_0-\hat{x}_0\right)}\right)\}
\end{gather}
Linearizing $f$ around $\hat{x}_0$, h around $\hat{x}_1 = f(\hat{x}_0, u_0)$: 
\begin{gather}
	\approx  \exp 
			\{-\frac{1}{2}\left(\nrmsq{}{\Sigma_w^{-\frac{1}{2}}\left(\hat{x}_1+H_f(x_0-\hat{x}_0)-x_1\right)}+
			\nrmsq{}{\Sigma_0^{-\frac{1}{2}}\left(x_0-\hat{x}_0\right)}\right)\}
			 = \\
			=  \exp 	
			\{-\frac{1}{2}\left(\nrmsq{}{\Sigma_w^{-\frac{1}{2}}
			\begin{pmatrix}
			H_f & -I
			\end{pmatrix} \begin{pmatrix}
			x_0 \\ x_1
			\end{pmatrix}
			+ 
			\Sigma_w^{-\frac{1}{2}}\left(\hat{x}_1-H_f\hat{x}_0\right)} \right. 
			\\ \left.
			+ \nrmsq{}{ 
			\begin{pmatrix}
				\Sigma_0^{-\frac{1}{2}} & 0 
			\end{pmatrix}
			\begin{pmatrix}
			x_0 \\ x_1
			\end{pmatrix}			
			-\Sigma_0^{-\frac{1}{2}}\hat{x}_0}
			\right) \}
			\label{eq:integrand-gaussian}
\end{gather}
 this term is a 
Gaussian over $x_{0:1}$, and its marginalization w.r.t. $x_0$ yields a Gaussian 
in $x_1$ only, and everything collapses to a (linear, if we also marginalize 
$h$ around $\hat{x}_1$) least-squares problem, of which the solution may be 
refined by repeated 
re-linearization as in the previous problem. 

Could probably have a somewhat easier derivation if had marginalized the entire 
posterior w.r.t. $x_0$. 

\section{}
Joint covariance $\Sigma_{0:1}$ is of the same size as $x_{0:1}^{ } x_{0:1}^T$ 
(that is, square, 
with side the sum of vector dimensions). If $x_0\in \mathbb{R}^n$ and $x_1\in 
\mathbb{R}^m$ then $\Sigma_{0:1}\in \mathbb{R}^{n+m\times n+m}$, 
$\Sigma_{00}\in\mathbb{R}^{n\times n}$, $\Sigma_{11}\in\mathbb{R}^{m\times m}$, 
$\Sigma_{01}\in\mathbb{R}^{n\times m}$. 

Marginalization in covariance form: $\Sigma_1' = \Sigma_{11}$.  Marginalization 
in information form: $I_1' = I_{11} - I_{01}^TI_{00}^{-1}I_{01}$

We hence continue from \Eqref{eq:integrand-gaussian}. Denoting the integrand's 
information matrix as $I_{0:1}^-$, we get 
\begin{gather}
	I_{0:1}^- = 
	\begin{pmatrix}
	H_f^T\Sigma_w^{-1}H_f + \Sigma_0^{-1} & -H_f^T\Sigma_w^{-1} \\
	-\Sigma_w^{-1}H_f & \Sigma_w^{-1}
	\end{pmatrix}
\end{gather}
Linearizing the "innovation" term around $\hat{x}_1$ we get (denote $H_h$ the 
Jacobian of $h$ at $\hat{x}_1$)
\begin{gather}
	\nrmsq{_{\Sigma_v}}{z_1-h(x_1)} \approx
	\nrmsq{_{\Sigma_v}}{h(\hat{x}_1) + H_h (x_1-\hat{x}_1) - z_1} =
	\\
	= \nrmsq{}{
	\Sigma_v^{-\frac{1}{2}}\begin{pmatrix}
	0 & H_h
	\end{pmatrix}
	\begin{pmatrix}
	x_0 \\ x_1
	\end{pmatrix}
	+ \Sigma_v^{-\frac{1}{2}}(h(\hat{x}_1) - H_h \hat{x}_1 - z_1)}
\end{gather}
This means that the information matrix of the joint distribution is: 
\begin{gather}
	I_{0:1} = 	
	\begin{pmatrix}
		H_f^T\Sigma_w^{-1}H_f + \Sigma_0^{-1} & -H_f^T\Sigma_w^{-1} \\
		-\Sigma_w^{-1}H_f & H_h^T\Sigma_v^{-1}H_h + \Sigma_w^{-1}
		\end{pmatrix}
\end{gather}
From here, using marginalization formula: 
\begin{gather}
	I_1^\prime = 
	H_h^T\Sigma_v^{-1}H_h + \Sigma_w^{-1} - 
	\Sigma_w^{-1}H_f\left(H_f^T\Sigma_w^{-1}H_f + \Sigma_0^{-1}\right)^{-1}
	H_f^T\Sigma_w^{-1}
\end{gather}
And $\Sigma_1^\prime$ is the inverse of the above. 
\end{document}