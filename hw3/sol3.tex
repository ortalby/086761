\documentclass[a4paper]{scrreprt}

\usepackage[margin=0.1in]{geometry}

\usepackage[english]{babel}
\usepackage[utf8]{inputenc}

\usepackage{color,xcolor,ucs}

%\usepackage{mathptmx}       % selects Times Roman as basic font
%\usepackage{subfig}
\usepackage{floatrow}
\usepackage{tabularx}
\usepackage{float}
\usepackage{amsfonts}
\usepackage{helvet}         % selects Helvetica as sans-serif font
\usepackage{courier}        % selects Courier as typewriter font
\usepackage{type1cm}        % activate if the above 3 fonts are
% not available on your system
\usepackage{amsmath}
\usepackage{mathtools}
% using for \triangleq
% from http://tex.stackexchange.com/questions/151670/latex-symbol-for-leftrightarrow-with-triangleq
\usepackage{amssymb}
%
\usepackage{makeidx}         % allows index generation
\usepackage{comment}         % allows index generation
\usepackage{graphicx}        % standard LaTeX graphics tool
% when including figure files
\usepackage{epstopdf}
\epstopdfsetup{outdir=./figures/}

\usepackage{multicol}        % used for the two-column index
\usepackage[bottom]{footmisc}% places footnotes at page bottom
\usepackage{bm}

\usepackage{cite}
\usepackage{url}

%\usepackage[bookmarks=true]{hyperref}


\usepackage[unicode=true, bookmarks=true,colorlinks=true]{hyperref}
\usepackage{xr-hyper}


%\newfloat{algorithm}{t}{lop}
%\usepackage[linesnumbered, inoutnumbered]{algorithm2e}
\usepackage{algpseudocode,algorithm,algorithmicx}



\usepackage{xspace}
\usepackage{rotating}

\usepackage{tikz}
%\usepackage{pgfplots}
\usepackage{tkz-graph}
\usetikzlibrary{arrows,shapes,shadows,positioning,calc}


\usepackage{graphicx}
%\graphicspath{{figures/}}

\usepackage{threeparttable}
\usepackage{multirow}
%\usepackage[font=scriptsize,labelfont=bf]{caption}

\usepackage{blkarray}

\usepackage{tikz}
\usetikzlibrary{bayesnet}
\usepackage{subcaption}
\usepackage{verbatim}
\graphicspath{{images/}}

%\usepackage{setspace}
%\usepackage{titlesec}
%\titlespacing*{\section}
%{0pt}{5.5ex plus 1ex minus .2ex}{4.3ex plus .2ex}
%\titlespacing*{\subsection}
%{0pt}{5.5ex plus 1ex minus .2ex}{2.3ex plus .2ex}
\DeclareMathOperator*{\argmin}{arg\,min}
\DeclareMathOperator*{\argmax}{arg\,max}

%\theoremstyle{definition}
\newtheorem{example}{Example}[section]


\definecolor{vadim}{rgb}{1,0.6,0.7}
\definecolor{yuri}{rgb}{0.6,0.6,1}

\definecolor{marker}{rgb}{1,1,0}

\newcommand{\myputtext}[2]{
	\begin{tikzpicture}
	\node [fill=#2, rounded corners=2pt] {#1};
	\end{tikzpicture}
}
%\newcommand{\VI}[1]{\colorbox{vadim}{VI: #1}}
%\newcommand{\YF}[1]{\colorbox{yuri}{YF:#1}}
\newcommand{\VI}[1]{{\color{blue} #1}}
\newcommand{\YF}[1]{{\color{magenta} #1}}


\newcommand{\hl}[1]{\colorbox{marker}{#1}}

\newcommand{\Eqref}[1]{Eq.~(\ref{#1})}
\newcommand{\Figref}[1]{Fig.~\ref{#1}}


\newcommand{\Zeta}{\mathrm{Z}}
\newcommand{\bydef}{\ensuremath{\overset{def}{=}}}
\newcommand{\maxim}[2]{\ensuremath{\underset{#2}{#1}}}
\newcommand{\fnorm}[1]{\ensuremath{{\parallel}#1{\parallel_{\cal F}}}} % this removes spacing, could be ugly for final version


%\DeclarePairedDelimiterX{\nrmB}[2]{\left\lVert}{\right\rVert^2_{#2}}{#1}
\newcommand{\nrm}[2]{\ensuremath{{\parallel}{#2}{\parallel_{#1}}}}
%\newcommand{\nrmsq}[2]{\ensuremath{{\parallel}{#2}{\parallel^2_{#1}}}}
\DeclarePairedDelimiterX{\nrmsq}[2]{\lVert}{\rVert^2_{#1}}{#2}
\newcommand{\expt}[2]{\ensuremath{\underset{{#1}}{\mathbb E}{\{{#2}\}}}}
%\newcommand{\alias}[2]{{#1}\odot{#2}}
\newcommand{\alias}[1]{\ensuremath{\{{#1}\}_{\textbf{aliased}}}}% \xspace}
%\newcommand{\blf}[1]{\ensuremath{b[#1]}}

\newcommand{\prob}[1]{\ensuremath{\mathbb{P}({#1})}}
\newcommand{\blf}[1]{\prob{#1}}


%\usepackage{textcomp,    % for \textlangle and \textrangle macros
%            xspace}
%\newcommand\la{\textlangle\xspace}  % set up short-form macros
%\newcommand\ra{\textrangle\xspace}

%FIXME!
\newcommand\la{\langle\xspace}  % set up short-form macros
\newcommand\ra{\rangle\xspace}

\newcommand*\Let[2]{\State #1 $\gets$ #2}
\algrenewcommand\algorithmicrequire{\textbf{Input:}}
\algrenewcommand\algorithmicensure{\textbf{Input:}}
\algnewcommand{\LineComment}[1]{\State \(\triangleright\) #1}

%new formulation
\newcommand{\priorB}{\ensuremath{\blf{X^-_{k+1}}}\xspace}
\newcommand{\condB}{\blf{X_{k+1} \mid z_{k+1}, \his}\xspace}
\newcommand{\condBi}[1]{\blf{X_{k+1} \mid #1, z_{k+1}, \his}\xspace}
\newcommand{\event}[1]{\ensuremath{A_{#1}}\xspace}

%spaces
\newcommand{\poses}{\ensuremath{{\cal X}}\xspace}
\newcommand{\relpose}{\ensuremath{x^{(rel)}}\xspace}
\newcommand{\relposes}{\ensuremath{{\cal X}^{(rel)}}\xspace}
\newcommand{\observations}{\ensuremath{{\cal Z}}\xspace}
\newcommand{\events}{\ensuremath{\{\event{\mathbb{N}}\}}\xspace}
\newcommand{\controls}{\ensuremath{{\cal U}}\xspace}
\newcommand{\landmarks}{\ensuremath{{\cal L}}\xspace}

\newcommand{\classif}{\ensuremath{{\cal S}}\xspace}
\newcommand{\classes}{\ensuremath{{\cal C}}\xspace}

%history
\newcommand{\his}{\ensuremath{{\cal H}}\xspace}

% history with relative poses
\newcommand{\hisrp}{\ensuremath{{H}}\xspace}

%------------------------------------------------
% Statistical Independence 
% from http://jblevins.org/log/latex-tips
\newcommand\independent{\protect\mathpalette{\protect\independenT}{\perp}}
\def\independenT#1#2{\mathrel{\rlap{$#1#2$}\mkern2mu{#1#2}}}
%------------------------------------------------

\title{086761 - Homework 3}
\author{Yuri Feldman, 309467801 yurif@cs.technion.ac.il \\
	    Alexander Shender, 328626114 aka.sova@gmail.com }
\begin{document}
\maketitle
\chapter{}
\section{}
\begin{gather}
	\prob{x\mid z_1, z_2} \propto \prob{z_1, z_2\mid x} \prob{x} 
	= \prob{z_1\mid x} \prob{z_2\mid x} \prob{x} \\
	\propto e^{-\frac{1}{2}\left(
	\nrmsq{\Sigma_{_{v_1}}}{z_1-h_1(x)}
	+\nrmsq{\Sigma_{_{v_2}}}{z_2-h_2(x)}
	+ \nrmsq{\Sigma_0}{x-\hat{x}_0}
	\right)}
\end{gather}
Linearize the exponent around $\bar{x}$, denote $H_1$, $H_2$ - the Jacobians of 
$h_1$, $h_2$ respectively at $\bar{x}$: 
\begin{gather}
	\nrmsq{\Sigma_{_{v_1}}}{z_1-h_1(x)}
	+\nrmsq{\Sigma_{_{v_2}}}{z_2-h_2(x)}
	+ \nrmsq{\Sigma_0}{x-\hat{x}_0}
	\approx \\
	\nrmsq{\Sigma_{_{v_1}}}{H_1x+h_1(\bar{x})-H_1\bar{x}-z_1}
	+\nrmsq{\Sigma_{_{v_2}}}{H_2x+h_2(\bar{x})-H_2\bar{x}-z_2}
	+ \nrmsq{\Sigma_0}{x-\hat{x}_0}	
\end{gather}
The information matrix is the sum of quadratic coefficients (similar to the 
previous homework): 
\begin{gather}
	I = H_1^T\Sigma^{-1}_{v_1}H_1 + H_2^T\Sigma^{-1}_{v_2}H_2 + \Sigma_0^{-1} \\
	\mu = -I^{-1} 
	\left(
	H_1^T\Sigma^{-1}_{v_1}(h_1(\bar{x})-H_1\bar{x}-z_1)
	+H_2^T\Sigma^{-1}_{v_2}(h_2(\bar{x})-H_2\bar{x}-z_2)
	-\Sigma_0^{-1}\hat{x}_0
	\right)
\end{gather}
similar to the previous homework, this can be refined by updating linearization 
point $\bar{x}\gets \mu$ and recalculating Jacobians, until convergence. 

\section{}
Re-projection error for measurement $z$ given belief over state 
variables $x$ 
and $l$ is defined as
\begin{gather}
	e = z - \pi\left(\hat{x}, \hat{l}\right),
\end{gather}
where 
\begin{gather}
	\hat{x},\hat{l} = \argmax_{x,l}\,\blf{x,l\mid z_1, z_2}
\end{gather}

%Note: wish this had been stated in general at the lectures or clarified as 
%part 
%of the 
%exercise. At 
%the lectures, reprojection error was handled only as a constraint for 
%inferring 
%state. Re-projection error was not defined in the presence of belief. 


Joint state belief (prior to measurement $z$): 
\begin{gather}
	\prob{x, l \mid z_1, z_2} = \prob{x\mid l, z_2, z_2}\prob{l\mid z_2, z_2} = 
	\prob{x\mid z_1, z_2}\prob{l}, 
\end{gather}
whence
\begin{gather}
	\hat{x} = \mu, \enspace \hat{l} = \hat{l}_0 \\
	e = z-\pi(\mu, \hat{l}_0)
\end{gather}

\section{}
\begin{gather}
	\prob{x,l\mid z_1, z_2, z} = \frac{\prob{z\mid x, l, z_1, z_2} \prob{x, 
	l\mid z_1, z_2}}{\prob{z\mid z_1, z_2}} = \\
	= \frac{\prob{z\mid x, l}\prob{x\mid z_1, 
	z_2}\prob{l}}{\int_{x,l}\prob{z\mid 
	x, l}\prob{x\mid z_1, z_2}\prob{l}dx dl} 
	\propto \prob{z\mid x, l}\prob{x\mid z_1, z_2}\prob{l}
\end{gather}

\section{}
From previous clause, 
\begin{gather}
	-\log (\prob{x,l\mid z, z_1, z_2}) \propto \nrmsq{\Sigma_v}{z-\pi(x,l)} + 
	\nrmsq{\Sigma}{x-\mu} + \nrmsq{\Sigma_{l0}}{l-\hat{l}_0} + C
\end{gather}
With $C$ constant (normalizing factor). Linearizing $\pi$ around $\mu, 
\hat{l}_0$, and denoting 
its Jacobian at that 
point as ${H_\pi = \begin{pmatrix}
H_\pi^{(x)} & H_\pi^{(l)}
\end{pmatrix}}$ we can rewrite
\begin{gather}
	\approx 
	\nrmsq{\Sigma_v}{\pi(\mu,\hat{l}_0)+H_\pi^{(x)}(x-\mu)+H_\pi^{(l)}(l-\hat{l}_0)-z}
	 + \nrmsq{\Sigma}{x-\mu} + \nrmsq{\Sigma_{l0}}{l-\hat{l}_0} + C
	 \label{eq:1d-joint-log-posterior}
\end{gather}
extracting the quadratic term coefficients (recall that variable is the joint 
state  
i.e. $(x^T\, l^T)^T$), we get that 
\begin{gather}
	= \begin{pmatrix}
	x^T & l^T
	\end{pmatrix}
	\left(
	H_\pi^T \Sigma_v^{-1} H_\pi
	+ \begin{pmatrix}
	\Sigma^{-1} & 0 \\ 0 & \Sigma^{-1}_{l0}
	\end{pmatrix}
	\right)
	\begin{pmatrix}
		x \\ l
		\end{pmatrix} + \dots , 
\end{gather}
and so
\begin{gather}
	I^\prime = H_\pi^T \Sigma_v^{-1} H_\pi
		+ \begin{pmatrix}
		\Sigma^{-1} & 0 \\ 0 & \Sigma^{-1}_{l0}
		\end{pmatrix}
		= \begin{pmatrix}
				H_\pi^{(x)T} \Sigma_v^{-1} H_\pi^{(x)} + \Sigma^{-1} & 
				H_\pi^{(x)T} \Sigma_v^{-1} H_\pi^{(l)} \\ 
				H_\pi^{(l)T} \Sigma_v^{-1} H_\pi^{(x)} & 
				H_\pi^{(l)T} \Sigma_v^{-1} H_\pi^{(l)} + \Sigma^{-1}_{l0}
				\end{pmatrix}
\end{gather}
The latter, similar to before, can be refined by iteritavely relinearizing 
around the minimum point 
of \Eqref{eq:1d-joint-log-posterior} (calculated e.g. using the linear term, or 
finding the derivative zero, or formulating as a linear min-squares solution).

To state explicitly, a min squares formulation of 
\Eqref{eq:1d-joint-log-posterior} is
\begin{gather}
\argmin_{x,l}
	\left\lVert 
	\underset{A}{\underbrace{
			\begin{pmatrix}
				\Sigma_v^{-\frac{1}{2}}H_\pi^{(x)}  
				& \Sigma_v^{-\frac{1}{2}} H_\pi^{(l)} \\
				\Sigma^{-\frac{1}{2}} & 0 \\
				0 & \Sigma^{-\frac{1}{2}}_{l0}
			\end{pmatrix}
			}}
	\begin{pmatrix}
	x \\ l
	\end{pmatrix}
	- 
	\begin{pmatrix}
		\Sigma_v^{-\frac{1}{2}}\left(-\pi(\mu,\hat{l}_0)+H_\pi^{(x)}\mu + 
		H_\pi^{(l)}\hat{l}_0 + z\right) \\
		\Sigma^{-\frac{1}{2}}\mu \\ \Sigma_{l0}^{-\frac{1}{2}}\hat{l}_0
	\end{pmatrix} \right\rVert^2, 
\end{gather}
whence we obtain the same result for the information matrix ($I^\prime=A^TA$). 

\chapter{}
\section{}
Denote camera calibration matrices as $K_1$, $K_2$ respectively. 
\begin{gather}
	t\doteq t^{C_2}_{C_1\to C_2} = R_G^{C_2}\cdot(t^G_{C_1\to G}-t^G_{C_2\to 
	G}) = R_2^T\cdot(t_1-t_2) \\
	R \doteq R_{C_1}^{C_2} = R_G^{C_2}\cdot R_{C_1}^{G} = R_2^T \cdot R_1
\end{gather}
The epipolar constraint: 
\begin{gather}
	\begin{pmatrix}
		z_2^T  & 1
	\end{pmatrix}
	K_2^{-T} [t]_\times R K_1^{-1} 
	\begin{pmatrix}
		z_1 \\ 1
	\end{pmatrix}
	= 0
\end{gather}

\section{}
\begin{gather}
	\prob{x_1, x_2\mid z_2, z_2} \propto \prob{z_1, z_2\mid x_1, x_2} 
	\prob{x_1} \prob{x_2}
\end{gather}

Treating the correspondence between $z_1$, $z_2$ as the measurement, we can 
model likelihood term as 
\begin{gather}
	\prob{z_1, z_2\mid x_1, x_2} = \prob{h(x_1, x_2, z_1, z_2)}
\end{gather}
where we assume that 
\begin{gather}
	\argmax_{x_1, x_2} 	\prob{z_1, z_2\mid x_1, x_2} = 
	\argmax_{x_1, x_2} \prob{h(x_1, x_2, z_1, z_2)}
	\intertext{and}
	h(x_1, x_2, z_1, z_2) \sim N(0, \Sigma_{ep})
\end{gather}
with the motivation being that given matching points $z_1$, $z_2$, epipolar 
constraint should satisfied, up to error, so $h$ should be close to 0. 

In summary, the MAP estimation can be written (in negative log-likelihood form, 
as before) as 
\begin{gather}
	\argmax_{x_1, x_2}\, \prob {x_1, x_2 \mid z_1, z_2} = \\
	= \argmin_{x_1, x_2}\, \nrmsq{\Sigma_{ep}}{h(x_1, x_2, z_1, z_2)} + 
	\nrmsq{\Sigma_{01}}{x_1-\mu_{01}} + 
	\nrmsq{\Sigma_{02}}{x_2-\mu_{02}}
\end{gather}
%Note: to our knowledge, residual error has not been mentioned in this course 
%up 
%to last lecture before submission (maybe I missed something?). Is this 
%intended 
%to penalize students who do not consult with other students when solving the 
%homework? 
\chapter{}
We start from the definition of fundamental matrix from the lecture: 
\begin{gather}
	F\doteq K_2^{-T} [t]_\times R K_1^{-1}, 
\end{gather}
with $K_{1,2}$, $t$ and $R$ defined as in question 2a. Denote $u = K_1 R^{-1}  
t$, where $R^{-1}$ exists since $R$ is a rotation matrix. Then
\begin{gather}
	F\,u = K_2^{-T} [t]_\times R K_1^{-1} K_1 R^{-1} t = K_2^{-T}\left(t\times 
	t\right) = 0, 
\end{gather}
whence $F$ is singular. 


\end{document}