\documentclass[a4paper]{scrreprt}
\usepackage{hyperref}
\usepackage{import}
\usepackage{appendix}
\usepackage{enumitem}
\usepackage{subcaption}
\usepackage{floatrow}
\usepackage{algorithm}
\usepackage{algpseudocode}
\import{./}{046201.sty}
\import{./}{jupyter-defs.sty}
%\import{G:/this_semester/latex/}{046201.sty}
%\graphicspath{{G:/this_semester/086761/hw4/}}

\usepackage{verbatim}

\usepackage{tikz}
\usetikzlibrary{bayesnet}

\usepackage{etoolbox}
%\makeatletter
%\patchcmd{\chapter}{\if@openright\cleardoublepage\else\clearpage\fi}{}{}{}
%\makeatother
\makeatletter
\patchcmd{\scr@startchapter}{\if@openright\cleardoublepage\else\clearpage\fi}{}{}{}
\makeatother

%\DeclareMathOperator*{\argmin}{arg\,min}
%\DeclareMathOperator*{\argmax}{arg\,max}

\newcommand{\group}[1]{%
  \par\noindent\textbf{#1:}
}

\renewcommand{\thesubsection}{\arabic{chapter}.\arabic{section}.\arabic{subsection}}

\title{086761 - Homework 1}
%\subtitle{Efficient Marginal Likelihood Optimization in Blind Deconvolution - 
%A.Levin, Y.Weiss, F.Durand, W.T.Freeman}
\author{Yuri Feldman, 309467801}
\begin{document}
\maketitle
%\tableofcontents
%\newpage
\chapter{}
\section{}
\begin{gather}
	f_x(x) = \frac{1}{\sqrt{\abs{2\pi\Sigma_x}}} \cdot 
	e^{-\frac{1}{2}(x-\mu_x)^T\Sigma_x^{-1} (x-\mu_x)}
\end{gather}
\section{}
$\mu_y$, $\Sigma_y$ can be calculated using expectation properties (and do not 
depend on $y$ being Gaussian):
\begin{gather}
	\mu_y = \expect Ax + b = A\expect x + b = \boxed{A\mu_x + b} \\
	\Sigma_y = \expect\left[ (Ax+b-\mu_y)(Ax+b-\mu_y)^T\right] = 
	\expect\left[ A(x-\mu_x)(x-\mu_x)^T A^T \right] = \\
	A\expect\left[ 
	(x-\mu_x)(x-\mu_x)^T\right] A^T = \boxed{A\Sigma_x A^T}
\end{gather}
Next, we show that $y$ is Gaussian. In the case where $A$ is invertible this 
can be directly obtained from random vector transformation formula (1-to-1 
transformation). Since $y=Ax+b$ the Jacobian matrix is $A$ and since the 
mapping is one-to-one (A is invertible) according to the transformation 
theorem: 
\begin{gather}
	f_Y(y) = \frac{1}{\abs{A}} f_X(A^{-1}(y-b))
\end{gather}
ignoring the constant normalization factors (since they don't depend on $x$ and 
only guarantee the distributions' summing to 1) we calculate the exponent in 
the 
right hand side: 
\begin{gather}
	-\frac{1}{2}\left(A^{-1}(y-b)-\mu_x\right)^T\Sigma_x^{-1}\left(A^{-1}(y-b)-\mu_x\right)
	 = \\
	-\frac{1}{2}\left(y-b-A\mu_x\right)^TA^{-T}\Sigma_x^{-1}A^{-1}\left(y-b-A\mu_x\right)
	 = \\
		-\frac{1}{2}\left(y-(A\mu_x+b)\right)^T\left(A\Sigma_xA^T\right)^{-1}\left(y-(A\mu_x+b)\right),
\end{gather}
which exactly gives a Gaussian distribution (in particular, with the mean and 
variance we calculated beforehand). 

We now deal with the case where $A\in \mathbb{R}^{m\times n}$ is not invertible 
- either rectangular or 
square with zero determinant. 
Consider the SVD decomposition of $A$: 
\begin{gather}
	A = U \Sigma V^T
\end{gather}
Here, $U$ and $V$ are square invertible matrices of respective sizes $m\times 
m$ and $n\times n$. Note that $V^T=V^\ast$ as $A$ is real. 
$\Sigma$ is a rectangular matrix the same size as $A$ with 
the main diagonal containing the singular values (and zero elsewhere). We can 
now write: 
\begin{gather}
	y = U\Sigma V^Tx + b
\end{gather}
Here, $V^Tx$ is Gaussian, since $V$ is invertible. Without loss of generality 
denote 
\begin{gather}
	\Sigma = \begin{pmatrix}
	\Sigma_1 & 0 \\ 0 & 0
	\end{pmatrix} \qquad V^Tx = \begin{pmatrix}
	v_1 \\ v_2
	\end{pmatrix},
\end{gather}
where $\Sigma_1$ is (square) diagonal, with positive entries (and in 
particular, is 
invertible), and has the same number of columns as the length of vector $v_1$. 
We have then that 
\begin{gather}
	\Sigma V^T x = \begin{pmatrix}
	\Sigma_1 v_1 \\ 0
	\end{pmatrix}, 
\end{gather}
where $\Sigma_1 v_1$ is Gaussian since $v_1$ is and $\Sigma_1$ is invertible, 
and hence $\Sigma V^T x =U^T (y-b)$ is (possibly degenerate) Gaussian (and 
hence, 
so is $y$). 

\chapter{}
\section{}
\begin{gather}
	p(x\mid z) = \frac{p(z\mid x) p(x)}{p(z)} = \frac{p(z\mid x) p(x)}{\int_x 
	p(z\mid x) p(x)\cdot dx} \propto p(z\mid x)\cdot p(x)
\end{gather}

\section{}
We are interested in the MAP estimate: 
\begin{gather}
	x^{\ast}_{_{MAP}}\doteq \argmax_x p(x\mid z) = \argmax_x p(z\mid 
	x)\cdot p(x) 
	= \argmax_x \log 
	p(z\mid x) + \log p(x) = \\
	\argmin_x \nrm{z-Hx}^2_{R} + 
	\nrm{x-x_0}^2_{\Sigma_0}
\end{gather}
Develop the latter: 
\begin{gather}
	\nrm{z-Hx}^2_{R} + 
		\nrm{x-x_0}^2_{\Sigma_0} = 	\nrm{R^{-1/2}(Hx-z)}^2 + 
				\nrm{\Sigma_0^{-1/2}(x-x_0)}^2 = \\
			\nrm{\begin{pmatrix}
			R^{-1/2}(Hx-z) \\
			\Sigma_0^{-1/2}(x-x_0)
			\end{pmatrix}}^2  = 
			\nrm{\begin{pmatrix}
						R^{-1/2}H \\
						\Sigma_0^{-1/2}
						\end{pmatrix} x - \begin{pmatrix}
						R^{-1/2}z \\ \Sigma_0^{-1/2}x_0
						\end{pmatrix}}^2, 
\end{gather}
from where $x^{\ast}_{_{MAP}}$ can be obtained as the least-squares solution 
using the pseudoinverse 
$A^\dagger b=(A^TA)^{-1}A^Tb$, or equivalently proceed directly, noting that 
the 
function 
is convex 
and has a unique extremum which is the minimum. Develop the above into: 
\begin{gather}	
		= x^T\left(H^TR^{-1}H+\Sigma_0^{-1}\right)x 
		-2\left(z^TR^{-1}H+x_0^T\Sigma_0^{-1}\right)x + 
		\left(z^TR^{-1}z+x_0^T\Sigma_0^{-1}x_0\right)
		\label{eq:2b:quadratic-form}
\end{gather}
find the zero of the gradient: 
\begin{gather}
	\nabla (\cdot) = 
	2\left(H^TR^{-1}H+\Sigma_0^{-1}\right)x^\ast-2\left(\Sigma_0^{-1}x_0+H^TR^{-1}z\right)
	 =0 \\
	\Longrightarrow \boxed{x^\ast_{_{MAP}} = 
	\left(H^TR^{-1}H+\Sigma_0^{-1}\right)^{-1}\left(\Sigma_0^{-1}x_0+H^TR^{-1}z\right)}
\end{gather}
The associated covariance is
\begin{gather}
	\boxed{\Sigma=\left(H^TR^{-1}H+\Sigma_0^{-1}\right)^{-1}},
\end{gather}
 as can be seen from the quadratic term in 
Eq.(\ref{eq:2b:quadratic-form}). 



\chapter{Code}

\begin{Verbatim}[commandchars=\\\{\}]
{\color{incolor}In [{\color{incolor}2}]:} \PY{k+kn}{import} \PY{n+nn}{numpy} \PY{k}{as} \PY{n+nn}{np}
	\PY{k+kn}{from} \PY{n+nn}{numpy}\PY{n+nn}{.}\PY{n+nn}{linalg} \PY{k}{import} \PY{n}{norm}
	\PY{k+kn}{from} \PY{n+nn}{math} \PY{k}{import} \PY{n}{sin}\PY{p}{,} \PY{n}{cos}\PY{p}{,} \PY{n}{pi}\PY{p}{,} \PY{n}{asin}\PY{p}{,} \PY{n}{atan2}\PY{p}{,} \PY{n}{degrees}
	\PY{k+kn}{from} \PY{n+nn}{tabulate} \PY{k}{import} \PY{n}{tabulate}
	
	\PY{n}{np}\PY{o}{.}\PY{n}{set\PYZus{}printoptions}\PY{p}{(}\PY{n}{suppress}\PY{o}{=}\PY{k+kc}{True}\PY{p}{,}\PY{n}{precision}\PY{o}{=}\PY{l+m+mi}{11}\PY{p}{)}
\end{Verbatim}


\section{Rotations}\label{rotations}

\begin{Verbatim}[commandchars=\\\{\}]
{\color{incolor}In [{\color{incolor}19}]:} \PY{k}{def} \PY{n+nf}{wedge}\PY{p}{(}\PY{n}{v}\PY{p}{)}\PY{p}{:}
	   \PY{k}{return} \PY{n}{np}\PY{o}{.}\PY{n}{matrix}\PY{p}{(}\PY{p}{[}\PY{p}{[}\PY{l+m+mi}{0}\PY{p}{,} \PY{o}{\PYZhy{}}\PY{n}{v}\PY{p}{[}\PY{l+m+mi}{2}\PY{p}{]}\PY{p}{,} \PY{n}{v}\PY{p}{[}\PY{l+m+mi}{1}\PY{p}{]}\PY{p}{]}\PY{p}{,}
	   \PY{p}{[}\PY{n}{v}\PY{p}{[}\PY{l+m+mi}{2}\PY{p}{]}\PY{p}{,} \PY{l+m+mi}{0}\PY{p}{,}  \PY{o}{\PYZhy{}}\PY{n}{v}\PY{p}{[}\PY{l+m+mi}{0}\PY{p}{]}\PY{p}{]}\PY{p}{,}
	   \PY{p}{[}\PY{o}{\PYZhy{}}\PY{n}{v}\PY{p}{[}\PY{l+m+mi}{1}\PY{p}{]}\PY{p}{,} \PY{n}{v}\PY{p}{[}\PY{l+m+mi}{0}\PY{p}{]}\PY{p}{,} \PY{l+m+mi}{0}\PY{p}{]}\PY{p}{]}\PY{p}{)}\PY{p}{;}
	
	\PY{k}{def} \PY{n+nf}{get\PYZus{}rotation\PYZus{}rodriguez}\PY{p}{(}\PY{n}{v}\PY{p}{,} \PY{n}{theta}\PY{o}{=}\PY{k+kc}{None}\PY{p}{)}\PY{p}{:}
	   \PY{k}{if} \PY{n}{theta} \PY{o+ow}{is} \PY{k+kc}{None}\PY{p}{:}
	   \PY{n}{theta} \PY{o}{=} \PY{n}{norm}\PY{p}{(}\PY{n}{v}\PY{p}{)}\PY{p}{;} 
	   \PY{n}{n} \PY{o}{=} \PY{n}{v}\PY{o}{/}\PY{n}{theta}\PY{p}{;} 
	   \PY{k}{else}\PY{p}{:} 
	   \PY{n}{n} \PY{o}{=} \PY{n}{v}\PY{p}{;} 
	
	   \PY{n}{cM} \PY{o}{=} \PY{n}{wedge}\PY{p}{(}\PY{n}{n}\PY{p}{)}\PY{p}{;} 
	   \PY{k}{return} \PY{n}{np}\PY{o}{.}\PY{n}{identity}\PY{p}{(}\PY{l+m+mi}{3}\PY{p}{)} \PY{o}{+} \PY{n}{sin}\PY{p}{(}\PY{n}{theta}\PY{p}{)}\PY{o}{*}\PY{n}{cM} \PY{o}{+} \PY{p}{(}\PY{l+m+mi}{1}\PY{o}{\PYZhy{}}\PY{n}{cos}\PY{p}{(}\PY{n}{theta}\PY{p}{)}\PY{p}{)}\PY{o}{*}\PY{n}{cM}\PY{o}{*}\PY{n}{cM}
\end{Verbatim}


\subsection{}\label{a}

\begin{Verbatim}[commandchars=\\\{\}]
{\color{incolor}In [{\color{incolor}14}]:} \PY{k}{def} \PY{n+nf}{rotx}\PY{p}{(}\PY{n}{phi}\PY{p}{)}\PY{p}{:}
	    \PY{n}{v} \PY{o}{=} \PY{n}{phi}\PY{o}{*}\PY{n}{np}\PY{o}{.}\PY{n}{array}\PY{p}{(}\PY{p}{[}\PY{l+m+mi}{1}\PY{p}{,} \PY{l+m+mi}{0}\PY{p}{,} \PY{l+m+mi}{0}\PY{p}{]}\PY{p}{)}\PY{p}{;} 
	    \PY{k}{return} \PY{n}{get\PYZus{}rotation\PYZus{}rodriguez}\PY{p}{(}\PY{n}{v}\PY{p}{)}
	
	\PY{k}{def} \PY{n+nf}{roty}\PY{p}{(}\PY{n}{theta}\PY{p}{)}\PY{p}{:}
	    \PY{n}{v} \PY{o}{=} \PY{n}{theta}\PY{o}{*}\PY{n}{np}\PY{o}{.}\PY{n}{array}\PY{p}{(}\PY{p}{[}\PY{l+m+mi}{0}\PY{p}{,} \PY{l+m+mi}{1}\PY{p}{,} \PY{l+m+mi}{0}\PY{p}{]}\PY{p}{)}\PY{p}{;} 
	    \PY{k}{return} \PY{n}{get\PYZus{}rotation\PYZus{}rodriguez}\PY{p}{(}\PY{n}{v}\PY{p}{)}
	
	\PY{k}{def} \PY{n+nf}{rotz}\PY{p}{(}\PY{n}{psi}\PY{p}{)}\PY{p}{:}
	    \PY{n}{v} \PY{o}{=} \PY{n}{psi}\PY{o}{*}\PY{n}{np}\PY{o}{.}\PY{n}{array}\PY{p}{(}\PY{p}{[}\PY{l+m+mi}{0}\PY{p}{,} \PY{l+m+mi}{0}\PY{p}{,} \PY{l+m+mi}{1}\PY{p}{]}\PY{p}{)}\PY{p}{;} 
	    \PY{k}{return} \PY{n}{get\PYZus{}rotation\PYZus{}rodriguez}\PY{p}{(}\PY{n}{v}\PY{p}{)} 
	
	\PY{k}{def} \PY{n+nf}{angles2rot}\PY{p}{(}\PY{n}{phi}\PY{p}{,} \PY{n}{theta}\PY{p}{,} \PY{n}{psi}\PY{p}{)}\PY{p}{:} 
	    \PY{k}{return} \PY{n}{rotz}\PY{p}{(}\PY{n}{psi}\PY{p}{)}\PY{o}{*}\PY{n}{roty}\PY{p}{(}\PY{n}{theta}\PY{p}{)}\PY{o}{*}\PY{n}{rotx}\PY{p}{(}\PY{n}{phi}\PY{p}{)}
\end{Verbatim}


\subsection{}\label{b}

\begin{Verbatim}[commandchars=\\\{\}]
{\color{incolor}In [{\color{incolor}16}]:} \PY{n+nb}{print}\PY{p}{(}\PY{l+s+s1}{\PYZsq{}}\PY{l+s+s1}{R= }\PY{l+s+se}{\PYZbs{}n}\PY{l+s+s1}{\PYZsq{}} \PY{o}{+} \PY{n+nb}{str}\PY{p}{(}\PY{n}{angles2rot}\PY{p}{(}\PY{n}{phi}\PY{o}{=}\PY{n}{pi}\PY{o}{/}\PY{l+m+mi}{4}\PY{p}{,} \PY{n}{theta}\PY{o}{=}\PY{n}{pi}\PY{o}{/}\PY{l+m+mi}{5}\PY{p}{,} \PY{n}{psi}\PY{o}{=}\PY{n}{pi}\PY{o}{/}\PY{l+m+mi}{7}\PY{p}{)}\PY{p}{)}\PY{p}{)}
\end{Verbatim}


\begin{Verbatim}[commandchars=\\\{\}]
R= 
[[ 0.72889912554  0.06766479742  0.68126906578]
[ 0.35101931853  0.81741496597 -0.45674742629]
[-0.58778525229  0.57206140282  0.57206140282]]

\end{Verbatim}

\subsection{}\label{c}
Development here is based on \cite{slabaugh1999computing}. 

We first write the rotation matrix as the product of individual axis rotation matrices:
\begin{gather}
\left(\begin{array}{ccc} 
\cos\left(\psi\right)\,\cos\left(\theta\right) & 
\cos\left(\psi\right)\,\sin\left(\phi 
\right)\,\sin\left(\theta\right)-\cos\left(\phi 
\right)\,\sin\left(\psi\right) & \sin\left(\phi 
\right)\,\sin\left(\psi\right)+\cos\left(\phi 
\right)\,\cos\left(\psi\right)\,\sin\left(\theta\right)\\ 
\cos\left(\theta\right)\,\sin\left(\psi\right) & 
\cos\left(\phi \right)\,\cos\left(\psi\right)+\sin\left(\phi 
\right)\,\sin\left(\psi\right)\,\sin\left(\theta\right) & 
\cos\left(\phi 
\right)\,\sin\left(\psi\right)\,\sin\left(\theta\right)-\cos\left(\psi\right)\,\sin\left(\phi
\right)\\ -\sin\left(\theta\right) & 
\cos\left(\theta\right)\,\sin\left(\phi \right) & 
\cos\left(\phi \right)\,\cos\left(\theta\right) 
\end{array}\right)
\end{gather}
Given rotation matrix $R$, we will want to extract angles by demanding per-element equality. We treat separately the case where $\cos(\theta) = 0$.
For the case $\theta=\frac{\pi}{2}$ we obtain the matrix: 
\begin{gather}
\left(\begin{array}{ccc} 
0 &  \cos\left(\psi\right)\,\sin\left(\phi 
\right)-\cos\left(\phi 
\right)\,\sin\left(\psi\right) & \sin\left(\phi 
\right)\,\sin\left(\psi\right)+\cos\left(\phi 
\right)\,\cos\left(\psi\right)\\ 
0 &  \cos\left(\phi \right)\,\cos\left(\psi\right)+\sin\left(\phi 
\right)\,\sin\left(\psi\right) & \cos\left(\phi 
\right)\,\sin\left(\psi\right)-\cos\left(\psi\right)\,\sin\left(\phi
\right)\\
-1 &  	0 & 0 
\end{array}\right) \\
= 
\left(\begin{array}{ccc} 
0 &  \sin(\phi-\psi) & \cos(\phi-\psi) \\ 
0 &  \cos(\phi-\psi) & -\sin(\phi-\psi) \\
-1 &  	0 & 0 
\end{array}\right)
\end{gather}

For the case $\theta=-\frac{\pi}{2}$: 
\begin{gather}
\left(\begin{array}{ccc} 
0 &  -\cos\left(\psi\right)\,\sin\left(\phi 
\right)-\cos\left(\phi 
\right)\,\sin\left(\psi\right) & \sin\left(\phi 
\right)\,\sin\left(\psi\right)-\cos\left(\phi 
\right)\,\cos\left(\psi\right)\\ 
0 &  \cos\left(\phi \right)\,\cos\left(\psi\right)-\sin\left(\phi 
\right)\,\sin\left(\psi\right) & -\cos\left(\phi 
\right)\,\sin\left(\psi\right)-\cos\left(\psi\right)\,\sin\left(\phi
\right)\\
1 &  	0 & 0 
\end{array}\right) \\
= 
\left(\begin{array}{ccc} 
0 &  -\sin(\phi+\psi) & -\cos(\phi+\psi) \\ 
0 &  \cos(\phi+\psi) & -\sin(\phi+\psi) \\
1 &  	0 & 0 
\end{array}\right)
\end{gather}
From here we proceed extracting the angles (see code and \cite{slabaugh1999computing}). 
\begin{Verbatim}[commandchars=\\\{\},samepage=true]
{\color{incolor}In [{\color{incolor}17}]:} \PY{k}{def} \PY{n+nf}{rot2angles}\PY{p}{(}\PY{n}{M}\PY{p}{)}\PY{p}{:}
	   \PY{n}{sin\PYZus{}theta} \PY{o}{=} \PY{o}{\PYZhy{}}\PY{n}{M}\PY{p}{[}\PY{l+m+mi}{2}\PY{p}{,} \PY{l+m+mi}{0}\PY{p}{]}\PY{p}{;}     
	   \PY{n}{theta} \PY{o}{=} \PY{n}{asin}\PY{p}{(}\PY{n}{sin\PYZus{}theta}\PY{p}{)}\PY{p}{;}     

	   \PY{k}{if} \PY{n}{sin\PYZus{}theta}\PY{o}{==}\PY{l+m+mi}{1}\PY{p}{:}
	      \PY{n}{psi} \PY{o}{=} \PY{l+m+mi}{0}\PY{p}{;} \PY{c+c1}{\PYZsh{} arbitrary}
	      \PY{n}{phi} \PY{o}{=} \PY{n}{atan2}\PY{p}{(}\PY{n}{M}\PY{p}{[}\PY{l+m+mi}{0}\PY{p}{,}\PY{l+m+mi}{1}\PY{p}{]}\PY{p}{,}\PY{n}{M}\PY{p}{[}\PY{l+m+mi}{0}\PY{p}{,}\PY{l+m+mi}{2}\PY{p}{]}\PY{p}{)}\PY{p}{;}     
	   \PY{k}{elif}    \PY{n}{sin\PYZus{}theta}\PY{o}{==}\PY{o}{\PYZhy{}}\PY{l+m+mi}{1}\PY{p}{:}
	      \PY{n}{psi} \PY{o}{=} \PY{l+m+mi}{0}\PY{p}{;} \PY{c+c1}{\PYZsh{} arbitrary}
	      \PY{n}{phi} \PY{o}{=} \PY{n}{atan2}\PY{p}{(}\PY{o}{\PYZhy{}}\PY{n}{M}\PY{p}{[}\PY{l+m+mi}{0}\PY{p}{,}\PY{l+m+mi}{1}\PY{p}{]}\PY{p}{,}\PY{o}{\PYZhy{}}\PY{n}{M}\PY{p}{[}\PY{l+m+mi}{0}\PY{p}{,}\PY{l+m+mi}{2}\PY{p}{]}\PY{p}{)}\PY{p}{;}        
	   \PY{k}{else}\PY{p}{:}
	      \PY{c+c1}{\PYZsh{} another valid solution: pi\PYZhy{}theta.             }
	      \PY{n}{cos\PYZus{}theta} \PY{o}{=} \PY{n}{cos}\PY{p}{(}\PY{n}{theta}\PY{p}{)}\PY{p}{;} 
	      \PY{n}{phi} \PY{o}{=} \PY{n}{atan2}\PY{p}{(}\PY{n}{M}\PY{p}{[}\PY{l+m+mi}{2}\PY{p}{,}\PY{l+m+mi}{1}\PY{p}{]}\PY{o}{/}\PY{n}{cos\PYZus{}theta}\PY{p}{,} \PY{n}{M}\PY{p}{[}\PY{l+m+mi}{2}\PY{p}{,} \PY{l+m+mi}{2}\PY{p}{]}\PY{o}{/}\PY{n}{cos\PYZus{}theta}\PY{p}{)}\PY{p}{;}
	      \PY{n}{psi} \PY{o}{=} \PY{n}{atan2}\PY{p}{(}\PY{n}{M}\PY{p}{[}\PY{l+m+mi}{1}\PY{p}{,}\PY{l+m+mi}{0}\PY{p}{]}\PY{o}{/}\PY{n}{cos\PYZus{}theta}\PY{p}{,} \PY{n}{M}\PY{p}{[}\PY{l+m+mi}{0}\PY{p}{,} \PY{l+m+mi}{0}\PY{p}{]}\PY{o}{/}\PY{n}{cos\PYZus{}theta}\PY{p}{)}\PY{p}{;}

	   \PY{k}{return} \PY{n}{phi}\PY{p}{,} \PY{n}{theta}\PY{p}{,} \PY{n}{psi}
\end{Verbatim}


\subsection{}\label{d}

\begin{Verbatim}[commandchars=\\\{\}]
{\color{incolor}In [{\color{incolor}22}]:} \PY{n}{phi}\PY{p}{,} \PY{n}{theta}\PY{p}{,} \PY{n}{psi} \PY{o}{=} \PY{p}{(}\PY{n}{degrees}\PY{p}{(}\PY{n}{x}\PY{p}{)} \PY{k}{for} \PY{n}{x} \PY{o+ow}{in} \PY{n}{rot2angles}\PY{p}{(}
\PY{n}{np}\PY{o}{.}\PY{n}{mat}\PY{p}{(}\PY{l+s+s1}{\PYZsq{}}\PY{l+s+s1}{ 0.813797681 \PYZhy{}0.440969611 0.378522306;  }\PY{l+s+se}{\PYZbs{}}
\PY{l+s+s1}{                           0.46984631   0.882564119 0.0180283112; }\PY{l+s+se}{\PYZbs{}}
\PY{l+s+s1}{                          \PYZhy{}0.342020143  0.163175911 0.925416578}\PY{l+s+s1}{\PYZsq{}}\PY{p}{)}\PY{p}{)}\PY{p}{)}
\PY{n+nb}{print}\PY{p}{(}\PY{l+s+s1}{\PYZsq{}}\PY{l+s+s1}{phi=}\PY{l+s+s1}{\PYZsq{}} \PY{o}{+} \PY{n+nb}{str}\PY{p}{(}\PY{n}{phi}\PY{p}{)}\PY{p}{,} \PY{l+s+s1}{\PYZsq{}}\PY{l+s+s1}{theta=}\PY{l+s+s1}{\PYZsq{}} \PY{o}{+} \PY{n+nb}{str}\PY{p}{(}\PY{n}{theta}\PY{p}{)}\PY{p}{,} \PY{l+s+s1}{\PYZsq{}}\PY{l+s+s1}{psi=}\PY{l+s+s1}{\PYZsq{}} \PY{o}{+} \PY{n+nb}{str}\PY{p}{(}\PY{n}{psi}\PY{p}{)}\PY{p}{)}
\end{Verbatim}


\begin{Verbatim}[commandchars=\\\{\}]
phi=9.999999994217536 theta=19.999999980143038 psi=29.99999998990158

\end{Verbatim}

\section{3D Rigid Transformation}\label{d-rigid-transformation}
\begin{gather}
	l^C = R_G^C l^G + t^C_{C\to G} = R_G^C (l^G + t^G_{C\to G})
\end{gather}


\begin{Verbatim}[commandchars=\\\{\},samepage=true]
{\color{incolor}In [{\color{incolor}23}]:} \PY{n}{trans} \PY{o}{=} \PY{n}{np}\PY{o}{.}\PY{n}{mat}\PY{p}{(}\PY{p}{[}\PY{o}{\PYZhy{}}\PY{l+m+mf}{451.2459}\PY{p}{,} \PY{l+m+mf}{257.0322}\PY{p}{,} \PY{l+m+mi}{400}\PY{p}{]}\PY{p}{)}
\PY{n}{rot} \PY{o}{=} \PY{n}{np}\PY{o}{.}\PY{n}{mat}\PY{p}{(}\PY{l+s+s1}{\PYZsq{}}\PY{l+s+s1}{0.5363 \PYZhy{}0.8440 0; 0.8440 0.5363 0; 0 0 1}\PY{l+s+s1}{\PYZsq{}}\PY{p}{)}\PY{p}{;} 

\PY{n}{pt} \PY{o}{=} \PY{n}{np}\PY{o}{.}\PY{n}{mat}\PY{p}{(}\PY{p}{[}\PY{l+m+mi}{450}\PY{p}{,} \PY{l+m+mi}{400}\PY{p}{,} \PY{l+m+mi}{50}\PY{p}{]}\PY{p}{)}

\PY{n}{Tgl} \PY{o}{=} \PY{n}{rot}\PY{o}{.}\PY{n}{dot}\PY{p}{(}\PY{p}{(}\PY{n}{pt} \PY{o}{+} \PY{n}{trans}\PY{p}{)}\PY{o}{.}\PY{n}{T}\PY{p}{)}
\PY{n+nb}{print}\PY{p}{(}\PY{n}{Tgl}\PY{p}{)}

[[-555.20335297]
[ 351.31482926]
[ 450.        ]]

\end{Verbatim}


\section{Pose Composition}\label{pose-composition}
Commanded:
\begin{gather}
x^c_{k+1} = x_k + \begin{pmatrix} 1 \\ 0 \end{pmatrix} \\
R^c_{k+1} = R^c_k \\
\prescript{k+1}{k}T^{(commanded)} = \begin{pmatrix}
I & \begin{pmatrix}
1 \\ 0
\end{pmatrix} 
\end{pmatrix}
\end{gather}
Actual:
\begin{gather}
	x^a_{k+1} = x_k + \begin{pmatrix} 1.01 \\ 0 \end{pmatrix} \\
	R^a_{k+1} = R(\theta = 1^{\circ}) \cdot R^c_k \\
	\prescript{k+1}{k}T^{(actual)} = \begin{pmatrix}
	R(\theta^{\circ}) & \begin{pmatrix}
	1.01 \\ 0
	\end{pmatrix} 
	\end{pmatrix}
\end{gather}

\begin{Verbatim}[commandchars=\\\{\}]
{\color{incolor}In [{\color{incolor}24}]:} \PY{k}{for} \PY{n}{ii} \PY{o+ow}{in} \PY{n+nb}{range}\PY{p}{(}\PY{l+m+mi}{10}\PY{p}{)}\PY{p}{:} 

\end{Verbatim}

%\begin{gather}
%	\phi \psi
%\end{gather}

%\verbatiminput{G:/this_semester/086761/hw4/hw4.m}
\bibliographystyle{plain}
\bibliography{086761.bib}

\end{document}